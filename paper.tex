\documentclass{article}

\title{Controller parameter identification using direct collocation}


\begin{document}

\maketitle

\section{Introduction}

Given a mathematical model of a system operating inside a closed loop system
and experimental measurements of data collected from that system, it is common
to attempt to identify parameters of the mathematical model that cause the
models response to the measured inputs to best fit the measured output data.
For linear systems, there are a variety of solutions to these identification
problems, but for non linear systems the solution amounts to solving a
optimization problem which is most certinaly laden with many local optima.

We are particularly interested in identifying the parameters of the controller
in a closed loop system. It is theorized that creatures operate, at least
partially, under a closed loop during locomotion. Humans for example have
proprioceptive, vestibular, etc sensors that provide a rich set of internal
measurements that guide our choice for muscle activation. If the map from
sensastion to acuatation can be identified, then the engineer will have a
mathematical model of the highly evolved control system human's employ that can
be used for biomicry in robotic controls. This is potentially useful for
humanoid robots and assitive devices for both able body and disabled
locomoters.

In this paper, we present the use of direct collocation to identify the
controller parameters of controled inverted pendulum that is excited by
pseudo-random external inputs. The method is compared to more traditional
shooting optimization methods and also a direct identification method.

Direct collocation is more commonly used to find the open loop controls that
drive a mathematical model to track a particular trajectory.

\section{Closed Loop System Model Description}

Herein we make use of a classic planar n-link inverted pendulum on a cart as a
our plant model. The cart with mass, $m_o$ can move laterally but is restricted
to the origin by a linear spring and damper. The links are masseless and there
is a point mass at each link joint. The inputs to the open loop system are a
lateral force which acts on the cart and ``joint torques'' at each joint which
apply an equal and opposite torque between the adjacent joints.

\end{document}
