\documentclass{article}

% for images: png, pdf, etc
\usepackage{graphicx}

% for nice table formatting, i.e. /toprule, /midrule, etc
\usepackage{booktabs}

% for nice units
\usepackage{siunitx}

\usepackage{amsmath}

\title{Quiet Standing Controller Parameter Identification: A Comparison of
Methods}

\author{Jason K. Moore and Antonie van den Bogert}

\begin{document}

\maketitle

\section{Introduction}

It is hypothesized that a human operating during the quiet standing task uses
feedback to remain upright in the face of perturbations. For various reasons,
it is desirable to obtain mathematical models that predict a human's actuation
patterns given measured estimates of the sensory information available to the
human. Reasonably good models of the human's open loop musculoskeletal system
exist but models of the human's control system and the system process noises
are still less than adequate. The control model can possibly be derived from
first principles, but high level understanding of the human's sensory
neurological feedback patterns are difficult to derive from the low level
neurological first principles. These high level control descriptions may be
more easily arrived at through identification and learning techniques.

Here we present a numerical study comparing three methods of identifying the
controller parameters of a human quiet standing state feedback system. The
first method, direct identification, is by far the least computationally
intensive but suffers from bias due to the unknown processes in the modeled
system and neglect of the closed loop in the model~\cite{Kooij2010}. The second
method, single shooting, is a typical method for parameter identification but
is the most computationally intensive and often suffers from extreme
sensitivity to initial guesses. Finally, the third method, which has not been
used for control parameter identification in biological systems, is direct
collocation. We aim to show that direct collocation is better suited to control
parameter identification because it does not suffer from bias because it is
indirect, computation times are very low, and it is much less sensitive to
initial guesses than most single shooting methods.

\section{Musculoskeletal Model Description}
%
We make use of the widely used planar two link inverted pendulum model of human
for quiet standing. In particular, our model matches that described in
\cite{Park2004}. Figure~\ref{fig:free-body-diagram} shows the open loop system.
The human is modeled by two rigid bodies: the legs and the torso. These are
connected to each other at the hip joint, modeled as an ideal pin. The legs can
rotate about a pin joint relative to the ``platform'' and the platform/ankle
point can be laterally accelerated. The centers of mass of the legs and torso
are located on a line connecting the respective pin joints. The muscles are
modeled as simple joint torque actuators. The orientation of the bodies are
described by the generalized coordinates $\theta_a$ and $\theta_h$. Gravity $g$
acts on the bodies in the $-y$ direction.
%
\begin{figure}
  \centering
  \includegraphics{figures/free-body-diagram.pdf}
  \caption{Free body diagram of musculoskeletal model used in this study.}
  \label{fig:free-body-diagram}
\end{figure}

The equations of motion were formed symbolically using Kane's
Method~\cite{Kane1985} using the \verb|mechanics| package in
SymPy~\cite{Gede2013}. The model derivation is included in the
\verb|src/model.py| file and implemented in a class named
\verb|QuietStandingModel|. The non-linear equations of motion take this form:
% TODO : This is too long!
\begin{equation}
  \begin{bmatrix}
  0 \\
  0 \\
  0 \\
  0
\end{bmatrix}
=
\begin{bmatrix}
\omega_{a} - \dot{\theta}_{a}\\
\omega_{h} - \dot{\theta}_{h}\\
d_{L} g m_{L} \operatorname{sin}\left(\theta_{a}\right) + d_{L} m_{L} a \operatorname{cos}\left(\theta_{a}\right) + d_{T} g m_{T} \operatorname{sin}\left(\theta_{a} + \theta_{h}\right) + d_{T} l_{L} m_{T} \left(\omega_{a} + \omega_{h}\right)^{2} \operatorname{sin}\left(\theta_{h}\right) - d_{T} l_{L} m_{T} \omega^{2}_{a} \operatorname{sin}\left(\theta_{h}\right) + d_{T} m_{T} a \operatorname{cos}\left(\theta_{a} + \theta_{h}\right) + g l_{L} m_{T} \operatorname{sin}\left(\theta_{a}\right) + l_{L} m_{T} a \operatorname{cos}\left(\theta_{a}\right) - \left(I_{T} + d_{T} m_{T} \left(d_{T} + l_{L} \operatorname{cos}\left(\theta_{h}\right)\right)\right) \dot{\omega}_{h} - \left(I_{L} + I_{T} + d_{L}^{2} m_{L} + m_{T} \left(d_{T}^{2} + 2 d_{T} l_{L} \operatorname{cos}\left(\theta_{h}\right) + l_{L}^{2}\right)\right) \dot{\omega}_{a} + T_{h}\\
d_{T} g m_{T} \operatorname{sin}\left(\theta_{a} + \theta_{h}\right) - d_{T} l_{L} m_{T} \omega^{2}_{a} \operatorname{sin}\left(\theta_{h}\right) + d_{T} m_{T} a \operatorname{cos}\left(\theta_{a} + \theta_{h}\right) - \left(I_{T} + d_{T}^{2} m_{T}\right) \dot{\omega}_{h} - \left(I_{T} + d_{T} m_{T} \left(d_{T} + l_{L} \operatorname{cos}\left(\theta_{h}\right)\right)\right) \dot{\omega}_{a} + T_{a}
\end{bmatrix}
\end{equation}

The numerical values of the open loop model constants were estimated using
Yeadon's method~\cite{Yeadon1989} and the software package
\verb|yeadon|~\cite{Dembia2014}. The body geometry measurements are included in
\verb|raw-data/yeadon-measurements.yml| for a 28 year old male.
Table~\ref{tab:model-constants} reports the computed constants for the model.
%
\begin{table}
  \centering
  \caption{Constant parameters in the plant}
  
\begin{tabular}{llll}
  \toprule
  Variable & Description & Value & Units \\
  \midrule
$l_{L}$ & booger & 0.878 & \si{\meter} \\
$d_{L}$ & booger & 0.572 & \si{\meter} \\
$d_{T}$ & booger & 0.314 & \si{\meter} \\
$m_{L}$ & booger & 32.126 & \si{\kilogram} \\
$m_{T}$ & booger & 48.831 & \si{\kilogram} \\
$I_{L}$ & booger & 1.799 & \si{\kilogram\meter\squared} \\
$I_{T}$ & booger & 2.481 & \si{\kilogram\meter\squared} \\
$g$ & booger & 9.810 & \si{\meter\per\second\squared} \\
  \bottomrule
\end{tabular}
  \label{tab:model-constants}
\end{table}

\section{Control Model Description}

To close the loop, we assume the human can at least continuously sense the full
state. There are physiological reasons that back this assumption but we mostly
choose it for computational simplicity. Given the state vector
%
\begin{equation}
  \mathbf{x} = \left[ \theta_a \quad \theta_h \quad \omega_a \quad \omega_h \right]^T
\end{equation}
%
we close the loop with
%
\begin{equation}
  \mathbf{T} = \left[ T_a \quad T_h \right]^T = \mathbf{K} (\mathbf{x}_{r} - \mathbf{x})
\end{equation}
%
where $\mathbf{x}_r$ is the desired reference state and $\mathbf{K}$ is a
matrix of feedback gains
%
\begin{equation}
  \mathbf{K} =
  \begin{bmatrix}
    k_{00} & k_{01} & k_{02} & k_{03} \\
    k_{10} & k_{11} & k_{12} & k_{13} \\
  \end{bmatrix}
  .
\end{equation}

We also consider a process noise, in our case an additive noise to the state
error. This primarily represents the human's error in estimating the state and
change in the desired state, but can also account for modeling errors. With the
reference noise $\mathbf{x}_n$, plant input becomes
%
\begin{equation}
  \mathbf{T} = \mathbf{K} (\mathbf{x}_{r} + \mathbf{x}_n - \mathbf{x}).
\end{equation}

We choose a set of realistic numerical gain values based on those presented in
\cite{Park2004} that stabilize the non-linear model around the vertical
equilibrium point:
%
\begin{equation}
  \mathbf{K} =
  \begin{bmatrix}
    950.0 & 175.0 & 185.0 & 50.0 \\
    45.0 & 290.0 & 60.0 & 26.0
  \end{bmatrix}
\end{equation}

With the system closed the only inputs are then the acceleration of the
platform $a$ and the reference noise $\mathbf{x}_n$ and both of which are
treated as specified exogenous inputs.

\section{Data Measurement}
%
There are many likely measurements one can use for identification purposes and
the different identification methods we propose each require a minimal set of
measurements. To generate artificial measurements we simulate the closed loop
system by integrating the explicit first order form of the equations of motion
forward in time with the variable step integration routine available in
odepack's \verb|lsoda| routine and accessed through SciPy's integration
wrappers. We choose the following measurements with optionally additive
Gaussian measurement noise $v(t, 0, \sigma)$.
%
\begin{align}
  \mathbf{x}_m = \left[ \mathbf{\theta} \quad \mathbf{\omega} \right]^T +
    \left[\mathbf{v}_{\theta}(0, \sigma_\theta) \quad \mathbf{v}_{\omega}(0,
    \sigma_\omega)\right]^T \\
  \mathbf{T}_m = \mathbf{T} + \mathbf{v}_T(0, \sigma_T) \\
  a_m = a + v_a(0, \sigma_a)
\end{align}

We use a sum of sinusoids with a bandwidth designed to fall within the human's
operating bandwidth for the specified platform acceleration. It is made up of
twelve sinusoids with fixed frequencies, a fixed amplitude, and randomly
generated phase shifts between $0$ and $2\pi$. The 12 frequencies are
logarithmically spaced between 0.03~\si{\hertz} and 2.18~\si{\hertz}.


\section{Direct Identification}

The direct approach can be used to identify the gains in the controller. The
accuracy of this approach relies heavily on the ratio of the system's process
noise and the applied pertrubations \cite{Kooij2005}. To implement the inputs
to the controller (-x) and the outputs of the controller (T) are assumed to be
measured. A linear identifcation model is constructed and linear least squares
can be used to compute the optimal gains for a set of measurments.

For the identification we assume that the controller model is:

\begin{equation}
  u = -K * x
\end{equation}

i.e. not affected by reference noise or deviating reference

Measuered states and joint torques
$\mathbf{X}$ : N x n
$\mathbf{T}$ : N x m

Unknown gains
$\tilde{\mathbf{K}}$: n x m

\begin{equation}
  -x \tilde{K} = \mathbf{T}
\end{equation}

The least squares estimation


\section{Indirect Identification: Shooting}

Indirect identification is based on minimizing the following cost function:

\begin{align}
  J(p) = \int_{t_0}^{t_f} [x_m(t) - x(t, p)]^2 dt
\end{align}

the state at any time is determined by integrating the eqations of motion

\begin{equation}
  x = \int_{t_0}^{t_f} f(x, t, r_m, p_m, p) dt
\end{equation}

$r_m$ : measured exongenous input
$p_m$ : measured constant parameters
p : unknown constant parameters

given the initial state $x_0$ \footnote{Here we assume that the initial state is
zero and do not include it in the objective's unknowns}.

To test shooting we make use both a gradient based sovler and a gradient free
evolutionary algorithm.

The quasi-Newton method of Broyden, Fletcher, Goldfarb, and Shanno is a common
general purpose minimizer for unconstrained problems. And we use the CMAES
algorithm which is has been successfully used for control identification
purposes \cite{Wang2010}.

\begin{equation}
  J(\tilde{K}) = h \sum_1^N (\bar{x}_i - x_i)^2
\end{equation}

\section{Indirect Identification: Direct Collocation}

We make use of direct collocation to transform the parameter identifaction
problem into a large scale non-linear programming problem. Do do so we first
assume that the discrete integral can be described by backward Euler
integration, giving an approximation of the state derivative as

\begin{align}
  x_{i} = x_{i-1} + h f(x_{i}, t_{i}) \\
  \dot{x} \approx \frac{x_i - x_{i-1}}{h} =  f(t_i, x_i)
\end{align}

For $t_i$ where $i=2 \dots N$ and the above assumption, we form $N-1$ algebraic
equations which must be satisfied.

\begin{align}
  0 = \mathbf{c}(\mathbf{\theta}) \\
  0 = f_i(x_{i}, x_{i-1}, u_i, p, h)
\end{align}

The objective function is the linear least squares norm which minimizes the
error in the measured data with respect to the model's trajectory.

\begin{equation}
  J(\theta) = h \sum_{i=1}^N \left[y_{mi} - y_i(\theta)\right]^2
\end{equation}

The NLP problem can be formed

\begin{align}
  \min_{\theta \in \Re^{n}}  J(\theta) \\
  c(\theta) = 0 \\
  \theta^L \leq \theta \leq \theta^U
\end{align}

The goal is to estimate the uknown parameters $p$, the controller gains in our
case, given noisy measurements, $y_{m_i}$.

\bibliographystyle{unsrt}
\bibliography{references}

\end{document}
